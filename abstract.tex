\prefacesection{Abstract}

High-order methods are truer-to-the-flow-physics and more accurate per degree of freedom than low-order methods. Their high computational intensity relative to their communication requirements makes them prime candidates for implementation in the latest hyper-parallel computing architectures like Graphical Processing Units (GPUs). Why are they not more prevalent in the toolset of design teams? 

One of the main barriers to wide adoption of high-order numerical methods in industrial applications is the schemes' low \emph{robustness} relative to low-order methods \cite{vincent2011facilitating}. Their stability is highly dependent on the quality of the grid, even when the solution is relatively smooth. Aliasing, underresolution, and non-smoothness are the main causes of instabilities.

This dissertation proposes solutions from two fronts: the creation of a set of families of numerical schemes with guaranteed linear stability and tunable dispersion-dissipation properties, and the formulation of filters with spectral effectiveness and element-local stencil. 

To ease the adoption, or increase \emph{usability}, of high-order methods in industry and academia, a well-documented, validated, verified, and constantly improved source code is needed. This dissertation walks the reader through the efforts by the \gls{acl} to provide such code: \gls{hf} (hifiles.stanford.edu).
