\section{Background} 
High-order numerical methods for unstructured grids promise to offer better accuracy than low-order schemes for a comparable computational cost. Their relative lower dispersion and dissipation make them prime candidates for use in \gls{les}~\cite{lodato2014structural}. Below is a brief history on the developments in high-order numerical schemes that form the foundations for this chapter.

The work of Reed and Hill~\cite{reed73} in the 70's introduced the \gls{dg} method to solve \gls{pde}s in variational form. Cockburn and Shu formulated the \gls{dg} method for conservation laws and advanced its theoretical foundations \cite{cockburn1989tvbII,cockburn1989tvbIII,cockburn1990rungeIV,cockburn1989runge,cockburn2001runge}. As a way to reduce the computational cost of the original \gls{dg} scheme, researchers developed a nodal variant. Hesthaven and Warburton give a through exposition of this method in their book~\cite{hesthaven2007nodal}. Kopriva and Kolias~\cite{Kopriva96} developed a staggered grid method based on the differential form of the equations. This method was later named \gls{sd} and was thoroughly studied by Liu et al.~\cite{liu2006discontinuous} and Wang et al.~\cite{wang2007spectral}. Wang~\cite{wang2002spectral} has also introduced the popular spectral volume method.

Noting the similarities between nodal \gls{dg} and the \gls{sd} schemes, Huynh introduced the \gls{fr} approach to high-order methods \cite{huynh2007flux,huynh2009reconstruction}. With this approach, it is possible to analyze and implement multiple high-order schemes within a unifying framework, including the \gls{sd} method and a variant of nodal \gls{dg} for the linear advection equation. Furthermore, Huynh used the \gls{fr} approach to create a variety of new high-order schemes with different stability and accuracy properties. Vincent et al. \cite{vincent2011new}, building on Jameson's proof of stability of the \gls{sd} scheme \cite{jameson2010proof}, formulated a class of \gls{esfr} schemes. These schemes are provably stable for all orders of accuracy in the linear advection-diffusion equations \cite{castonguay2013energy}. Williams et al. and Castonguay et al. have proved this stability for all orders in tetrahedral \cite{williams2013tet} and triangular \cite{castonguay2012new,williams2013tri} meshes.

In the $p+1$th order \gls{fr} and \gls{esfr} schemes for the 1-dimensional linear advection equation, the solution is represented by a polynomial of degree $p$ and the flux is represented by a polynomial of degree $p+1$. In this article, we show that it is possible to represent the flux with a polynomial of degree $m$ -- where $m \ge p+1$--, evaluate such flux at the regular $p+1$ points, retain the provable linear stability, and obtain the expected $m+1$th order convergence (limited only by interpolation errors). An important part of the proof consists of ensuring that the flux polynomial representations have continuous arbitrary derivatives at the flux points connecting interfaces between elements, hence we call these schemes $C^m$ Flux Reconstruction (CMFR) schemes. This general framework recovers \gls{esfr} schemes and provides an arbitrary number of parameters that can be used to optimize the dispersion and dissipation properties of the schemes. The work by Asthana \cite{asthana2014high} shows how it is possible to optimize the dispersion and dissipation properties of \gls{esfr} schemes. His work inspired the formulation of a provably stable scheme with more optimizable parameters, and thus the creation of the \gls{cmfr} schemes presented here.

The article starts with the formulation of the \gls{cmfr} schemes. We then present the proof of stability for all orders of accuracy of the \gls{cmfr} schemes for the linear advection equation. The proof of stability requires the formulation of some ``correction functions", so the discussion continues with the general procedure for finding these. To illustrate the process of scheme creation, we show the development of the \gls{cmfr} scheme that has fluxes continuous in the $0$th and $1$st derivatives: the $C^{0,1}$\gls{fr} (or C1FR) scheme. We then perform numerical experiments that demonstrate the schemes' $p+1$th order convergence when we use an $p$th order polynomial to represent the flux. The discussion is followed by showing how the \gls{cmfr} schemes can preserve the energy of high-frequency waves better than the nodal \gls{dg} method at the price of exchanging flux derivative information across element interfaces and additional work in the simulation pre-processing stage. We conclude by showing potential avenues of future research.