\section{Introduction}
The \gls{fr} approach to high-order methods provides a unifying framework to analyze and implement a large set of high-order schemes, including the nodal \gls{dg} and \gls{sd} methods. The unification occurs through the formulation of flux correction functions. The main appeal of \gls{fr} is its differential formulation, which is ideal for highly-parallel computational architectures. \gls{esfr} provides the added benefit of guaranteeing linear stability while having variable dispersion and dissipation properties parameterized by a single constant. Asthana~\cite{asthana2014high} found optimal values for such constant and found that the \gls{fr} scheme could be optimized further if the scheme's formulation were not constrained by this parameter.

With the intention of providing a framework whereby more parameters are introduced while linear stability is guaranteed, we formulate the $C^m$ Flux Reconstruction (CMFR) set of families of schemes. The main difference between \gls{esfr} and \gls{cmfr} is that the flux correction functions in \gls{cmfr} are forced to be continuous among elements in an arbitrary number of derivatives, while \gls{esfr} requires $C^0$ and $C^{p+1}$ continuity only --where $p$ is the degree of the polynomial used to discretize the solution.

In this article we present the proof of linear stability of the \gls{cmfr} set of families of schemes, the derivation of C1FR --a \gls{fr} scheme with $C^1$ correction functions--, and promising results for energy preservation of underresolved wavenumbers with C1FR.