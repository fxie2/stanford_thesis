\section{Conclusions}

We have presented a natural extension of the \gls{fr} approach. The \gls{cmfr} schemes guarantee 1-D linear stability and introduce an arbitrary number of parameters that modify the scheme's dispersive and dissipative properties. The addition of these parameters require the representation of the reconstructed flux to be $p+1$ or higher, where $p$ is the order of the polynomial used to represent the conservative solution.

 We have shown the derivation of the \gls{c1fr} scheme, which has reconstructed fluxes continuous in the zeroth and first derivatives across elements. This scheme, with a particular selection of its free parameter, was able to preserve the energy of an advected and diffused wave with a medium wavenumber better than the nodal \gls{dg} scheme. Certainly, this one example cannot be said to be generalizable. Nevertheless, it calls for a Von Neumann analysis to assess the impact of the free parameter on the scheme's properties. As the general \gls{cmfr} schemes have arbitrarily many such parameters, this analysis should be generalizable.

A major complication with the general \gls{cmfr} schemes is that the correction functions do depend on the element's Jacobian, so they are not as general and element-agnostic as those in the original \gls{fr} schemes. However, as shown by Allaneau~\cite{allaneau2011connections}, it is possible to formulate some \gls{fr} schemes as filtered \gls{dg} schemes without the need to find the correction functions explicitly. A similar analysis with the \gls{fr} schemes could yield element-dependent filtered \gls{dg} schemes whose properties can be understood or optimized in the \gls{cmfr} framework.

\gls{c1fr} solutions to the 1-D Euler equations suggest that the scheme is more robust than regular \gls{dg} in problems likely to arise in engineering applications. This is not to mean that solutions cannot be obtained with \gls{dg}, but rather that the built-in robustness in \gls{c1fr} could allow it to provide good solutions to challenging flow problems without the need of tuning, limiting, or filtering. Certainly, performing any of the latter could enhance the accuracy and robustness of the scheme, as it does in \gls{dg}.

Future work should include numerical experiments in 2-D and 3-D using tensor product elements to assess the extent to which the stability guarantees in 1-D translate to other dimensions. In addition, it is still unclear if the degree of continuity of the corrected fluxes could be beneficial in the solution of high order partial differential equations like the heat equation.