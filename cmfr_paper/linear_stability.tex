\section{Linear stability of $C^m$ continuous flux reconstruction $(\frd = a\urd)$}

In this section we show that the \gls{cmfr} schemes are stable in the 1-D linear advection equation in the following Sobolev-type norm:
\begin{equation}\label{eq:norm}
\begin{split}
|| \uud ||^2_m &=\sum_{n=1}^{N}\int_{x_n}^{x_{n+1}} \l\{ \sum^m_{r=0}
\frac{c_r}{2} \l( \dd{^r \uud}{x^r} \r)^2 \r\} dx \\
&=\sum_{n=1}^{N}   \sum^m_{r=0}
c_r \l(  \frac{1}{J_n^{2r}}\r) \ioo \l\{ \frac{1}{2}\l(\ddxir{\urd}{r} \r)^2
\r\} J_n \cdot d\xi 
\end{split}
\end{equation}
where $c_r$ for $0 \le r \le m$ are arbitrary constants. %To ensure Eqn.~\eqref{eq:norm}
%is a norm, let $c_0=1$. 
It is possible to find ranges of values of each $c_r$ for which Eqn. \eqref{eq:norm} is a norm. As shown later in section \ref{sec:advDiff}, negative values of $c_r$ are possible given that $\uud$ is a polynomial. To establish stability, we need to show that
\begin{equation}
\label{eq:ddnorm}
\DD{}{t}||\uud||^2_m = \sum_{n=1}^{N}   \sum^m_{r=0}
c_r \l(  \frac{1}{J_n^{2r}}\r) \DD{}{t}\ioo \l\{ \frac{1}{2}\l(\ddxir{\urd}{r} \r)^2
\r\} J_n \cdot d\xi \le 0
\end{equation} 

To that end, consider the \gls{fr} scheme in element $n$ for linear advection:
\begin{equation}
\label{eq:scheme}
\DD{\urd}{t} = -\frac{1}{J_n} \left[ a\ddxi{\urd} + \sum_{i=0}^m
\left\{ \IL
\ddxi{\gL} + \IR \ddxi{\gR}\right\} \right]
\end{equation}

To express Eqn. \eqref{eq:norm} in known terms, differentiate Eqn.
\eqref{eq:scheme} $r$ times, where $0\le r\le m$, with respect to $\xi$;
multiply by $\ddxir{\urd}{r}$; and integrate from $-1$ to $1$ to obtain
\begin{equation}
\label{eq:normterm1}
 \DD{}{t}\ioo \l\{ \frac{1}{2}\l( \ddxir{\urd}{r} \r)^2  \r\} d\xi = -\frac{1}{J_n}
\l[\text{\ding{172}}+\text{\ding{173}}+\text{\ding{174}} \r]
\end{equation}
where
\begin{align}
\label{eq:exp1}
 \text{\ding{172}}&= a \ioo \frac{1}{2} \ddxi{}\l( \ddxir{\urd}{r}\r)^2 d\xi=\frac{a}{2}\l[ 
\l( \ddxir{\urd}{r}\r)^2\bigg|_{R,n} - \l( \ddxir{\urd}{r}\r)^2\bigg|_{L,n} \r]\\
\label{eq:normterm2}
\text{\ding{173}}&= \sum^m_{i=0} \IL \ioo \ddxir{\urd}{r}\cdot \ddxir{\gL}{r+1} d\xi\\
\text{\ding{174}}&= \sum^m_{i=0} \IR \ioo \ddxir{\urd}{r}\cdot \ddxir{\gR}{r+1} d\xi
\end{align}


It is possible to simplify Eqn. \eqref{eq:normterm2} further.
% \begin{equation}
%  \text{\ding{173}}= \sum^m_{i=0} \IL \ioo \ddxir{\urd}{r}\cdot \ddxir{\gL}{r+1}
% d\xi\\
% \end{equation}
Integrating by parts,
\begin{align}
 \text{\ding{173}}&=\sum^m_{i=0} \IL \l\{ \ioo \l( \ddxi{} \l[ \ddxir{\urd}{r}
\ddxir{\gL}{r} \r]
-\ddxir{\gL}{r} \ddxir{\urd}{r+1} \r) d\xi \r\}\\
\text{\ding{173}}&= \sum^m_{i=0} \IL \l[\ddxir{\urd}{r} \ddxir{\gL}{r}\r]^1_{-1}
- \sum^m_{i=0} \IL \ioo
\ddxir{\gL}{r} \ddxir{\urd}{r+1} d\xi
\end{align}
By using the boundary conditions on $\gL$ given in Eqn.
\eqref{eq:gconstraints},
\begin{equation}
\label{eq:exp2}
\text{\ding{173}}=-\IL[r] J_n^r \ddxir{\urd}{r}\bigg|_{L,n} - \sum^m_{i=0} \IL \ioo
\ddxir{\gL}{r} \ddxir{\urd}{r+1} d\xi 
\end{equation}
Proceeding similarly with term \ding{174}
\begin{equation}
\label{eq:exp3}
\text{\ding{174}}=\IR[r] J_n^r\ddxir{\urd}{r}\bigg|_{R,n} - \sum^m_{i=0} \IR \ioo
\ddxir{\gR}{r}
\ddxir{\urd}{r+1} d\xi
\end{equation}

Note the difference in signs between \ding{173} and \ding{174}.

By replacing the expression from Eqn.~\eqref{eq:normterm1} into Eqn.~\eqref{eq:ddnorm}, we
obtain
\begin{equation}
\begin{split}
\label{eq:newnorm}
 \DD{}{t}&||\uud||^2_m = \sum_{n=1}^{N}\bigg\{   
\sum^m_{r=0}c_r \l(  -\frac{1}{J_n^{2r}}\r)  \frac{a}{2}\l[\l(
\ddxir{\urd}{r}\r)^2\bigg|_{R,n} - \l(\ddxir{\urd}{r}\r)^2\bigg|_{L,n} \r]  \\
&+ \sum^m_{r=0} c_r \l( - \frac{1}{J_n^{2r}}\r)  \l[ -\IL[r] J_n^r\ddxir{\urd}{r}
\bigg|_{L,n} - \sum^m_{i=0} \IL \ioo \ddxir{\gL}{r} \ddxir{\urd}{r+1} d\xi\r]\\
&+ \sum^m_{r=0} c_r \l( - \frac{1}{J_n^{2r}}\r)  \l[ \IR[r] J_n^r\ddxir{\urd}{r}
\bigg|_{R,n} -\sum^m_{i=0} \IR \ioo \ddxir{\gR}{r} \ddxir{\urd}{r+1} d\xi\r]\bigg\}
\end{split}
\end{equation}

%Notes: $g$ can be of any order as long as boundary constraints are satisfied. In other words, as will be seen later, $g$ must be at least of order $2(s+1)$ where $s$ is the number of types of flux continuities desired. If $g$ is $\mathcal{O}(p+s) ,s>1 \rightarrow$  cannot include $p\th$ order terms (simply) unless $c^p$ continuous. 

To de-clutter the notation, let us define $$\ddxir{*}{r} = \dD{*}{r} $$ and re-arrange Eqn.~\eqref{eq:newnorm} to obtain
\begin{equation}
\label{eq:timenorm}
 \DD{}{t}||\urd||^2_m = \text{\textcircled{a}} + \text{\textcircled{b}} + \text{\textcircled{c}}
\end{equation}
where
\begin{equation}
\begin{split}
 \label{eq:simpNormA}
 \text{\textcircled{a}} = -\sum_{n=1}^{N}  \bigg\{& 
 \sum^m_{r=0} c_r  \frac{a}{2} 
 \l[ \urd^{{(r)}^2} \big|_{R,n} - \urd^{{(r)}^2} \big|_{L,n} \r] 
 \l( \frac{1}{J_n^{2r}} \r) \\
 +& \sum^m_{r=0} -c_r \IL[r] J_n^r\urd^{(r)} \big|_{L,n} \l(
\frac{1}{J_n^{2r}}\r)\\
 +& \sum^m_{r=0} c_r \IR[r] J_n^r\urd^{(r)} \big|_{R,n} \l( \frac{1}{J_n^{2r}}\r)
 \bigg\} 
\end{split}
\end{equation}
\begin{equation}\label{eq:b_def}
 \text{\textcircled{b}} = \sum_{n=1}^{N} \l\{ 
 \sum^m_{r=0} c_r \sum^m_{i=0} \IL \ioo \dD{\gL}{r} \hat{u}^{{ (r+1)}} d\xi \l(
\frac{1}{J_n^{2r}}\r) \r\}
\end{equation}
\begin{equation}\label{eq:c_def}
 \text{\textcircled{c}} = \sum_{n=1}^{N}  \l\{ 
 \sum^m_{r=0} c_r \sum^m_{i=0} \IR \ioo \dD{\gR}{r} \hat{u}^{{ (r+1)}} d\xi \l(
\frac{1}{J_n^{2r}}\r) \r\}
\end{equation}


We show stability of \gls{cmfr} in the two following steps:
\begin{enumerate}
 \item We show that for a selection of interface values $\IL $ and $\IR $, 
 \begin{equation}\label{eq:a_le_0}
\text{\textcircled{a}} \le 0
 \end{equation}
 \item We explain how to find functions $\gL $ and $\gR $, for $i = 0,\ldots,m $ satisfying
conditions \eqref{eq:gconstraints} such that
\begin{equation}\label{eq:bc_equal_0}
\begin{split}
\text{\textcircled{b}}&=0\\
\text{\textcircled{c}}&=0
\end{split}
\end{equation}  
 for arbitrary
$c_r, r=1,\ldots,m$.
\end{enumerate}

By showing that expressions \eqref{eq:a_le_0} and \eqref{eq:bc_equal_0} hold, we conclude, from Eqn. \eqref{eq:timenorm} that
\begin{equation}
\DD{}{t}||\urd||_m \le 0
\end{equation}

\subsection{Part 1.}

In this part of the proof, we aim to show that the term \textcircled{a} in
Eqn.~\eqref{eq:timenorm} is non-positive.
\\
Rearranging and factoring terms in \textcircled{a}, Eqn. \eqref{eq:simpNormA} becomes
\begin{equation}
\begin{split}
\text{\textcircled{a}} =& -\sum_{n=1}^{N}  \bigg\{
 \sum^m_{r=0} c_r \l( \frac{1}{J_n^{2r}}\r)
 \bigg( \frac{a}{2} 
 \l[ \hat{u}^{{(r)}^2} \big|_{R,n} - \hat{u}^{{ (r)}^2} \big|_{L,n} \r]\\
 &- \IL[r] J_n^r\cdot \dD{\urd}{r} \big|_{L,n}
 +   \IR[r] J_n^r\cdot \dD{\urd}{r} \big|_{R,n}  \bigg)
 \bigg\}
 \end{split}
 \label{eq:aterm}
\end{equation}

Recall the definition of the correction constants $\IL[r] $ and
$\IR[r]$ in element $n$ from Eqn. \eqref{eq:jump_def},
\begin{equation}
\begin{split}
 \IL[r]J_n^r &= \dD{f}{r} \big|^I_{L,n} - {f}^{ (r)} \big|_{L,n}\\
 \IR[r]J_n^r &= \dD{f}{r} \big|^I_{R,n} - {f}^{ (r)} \big|_{R,n}
\end{split}
\end{equation}

In the case of linear advection, $\hat{f}^\delta = a\urd$, so the correction constants become
\begin{equation}
\label{eq:lincor}
\begin{split}
 \IL[r]J_n^r &= \dD{\hat{f}}{r} \big|^I_{L,n} - a\urdrL\\
 \IR[r]J_n^r &= \dD{\hat{f}}{r} \big|^I_{R,n} - a\urdrR
\end{split}
\end{equation}

Let us introduce the following generalized Roe flux at the interfaces, so
\begin{equation}
\label{eq:ifluxdef}
\begin{split}
\dD{\hat{f}}{r} \big|^I_{L,n} = \frac{1}{2}& \l[a\urdrL[n] + 
a\urdrR[n-1] \r] \\
&- \frac{1-{\alpha_r}}{2} \l|A_{r_{L,n}}\r| \l[\urdrL - \urdrR[n-1] \r]\\
\dD{\hat{f}}{r} \big|^I_{R,n} = \frac{1}{2}& \l[a\urdrL[n+1] + 
a\urdrR[n] \r]\\
&- \frac{1-{\alpha_r}}{2} \l|A_{r_{R,n}}\r| \l[\urdrL[n+1] - \urdrR[n] \r]
\end{split}
\end{equation}

Where $A_{r_{L,n}}$ and $A_{r_{R,n}}$ are the Jacobian matrices corresponding to the
$r^{\text{th}}$ derivative of the flux at the left ($L$) and right ($R$) interfaces of element $n$.
Equivalently,
\begin{equation}
\begin{split}
 A_{r_{L,n}} &= \l[ \frac{d \big( \hat{f}^{(r)} \big)}{d \big( \hat{u}^{(r)} \big)}
\r]_{L,n}\\
 A_{r_{R,n}} &= \l[ \frac{d \big( \hat{f}^{(r)} \big)}{d \big( \hat{u}^{(r)} \big)}
\r]_{R,n}
\end{split}
\end{equation}

Numerically, $A_{r_{L,n}}$ and $A_{r_{R,n}}$ can be evaluated for the linear advection equation as
\begin{equation}
\begin{split}
 A_{r_{L,n}} &= \frac{a \urdrL[n] - a \urdrR[n-1]}{\urdrL[n] - \urdrR[n-1]} = a\\%\aL\\
 A_{r_{R,n}} &= \frac{a \urdrL[n+1] - a \urdrR[n]}{\urdrL[n+1] - \urdrR[n]} = a%\aR = \aL[n+1]
\end{split}
\end{equation}

% Note that in the case of linear advection, $\aR = \aL = a$ for all $n$ and $r$. We leave the
% subscripts to later show that \textcircled{a} is non-negative even when the flux in non-linear.

Note that $A_{r_{L,n}} = A_{r_{R,n-1}}$ even for non-linear fluxes by construction.

Plugging these values of $A_{r_{L,n}}$ and $A_{r_{R,n}}$ into Eqn. \eqref{eq:ifluxdef}, and
substituting the updated interface flux values $\dD{\hat{f}}{r} \big|^I_{L,n}$ and $\dD{\hat{f}}{r}
\big|^I_{R,n}$ into the definition of the interface correction values in Eqn. \eqref{eq:lincor},
we obtain

\begin{equation}
\begin{split}
 \IL[r] J_n^r = \frac{1}{2} &\l\{a\urdrL[n] + 
a\urdrR[n-1] \r\}\\
 &- \frac{1-{\alpha_r}}{2} \l|a\r| \l\{\urdrL - \urdrR[n-1] \r\} - a\urdrL\\
 \IR[r] J_n^r = \frac{1}{2} &\l\{a\urdrL[n+1] + 
a\urdrR[n] \r\} \\
&- \frac{1-{\alpha_r}}{2} \l|a\r| \l\{\urdrL[n+1] - \urdrR[n] \r\} - a\urdrR
\end{split}
\end{equation}

Simplifying,
\begin{equation}
\label{eq:lincorUpdated}
\begin{split}
 \IL[r] J_n^r= \frac{a}{2}& \l\{-\urdrL[n] + \urdrR[n-1] \r\} \\
 &- \frac{1-{\alpha_r}}{2} \l|a\r| \l\{\urdrL
- \urdrR[n-1] \r\} \\
 \IR[r] J_n^r= \frac{a}{2}& \l\{\mbox{\;\;\:} \urdrL[n+1] - \urdrR[n] \r\} \\
 &- \frac{1-{\alpha_r}}{2}
\l|a\r|\l\{\urdrL[n+1] - \urdrR[n] \r\} 
\end{split}
\end{equation}


Using the updated values of $\IL[r]$ and $\IR[r]$ from Eqn. \eqref{eq:lincorUpdated},
Eqn. \eqref{eq:aterm} becomes
\begin{equation}
\begin{split}
 \text{\textcircled{a}} = -\sum_{n=1}^{N}  \bigg\{
 \sum^m_{r=0} & c_r \l( \frac{1}{J_n^{2r}}\r)
 \bigg( \frac{a}{2} 
 \l[ \hat{u}^{{ (r)}^2} \big|_{R,n} - \hat{u}^{{ (r)}^2} \big|_{L,n} \r] \\
 %
 &- \bigg[ \frac{a}{2} \l\{-\urdrL[n] + \urdrR[n-1] \r\} \\
 &\mbox{\;\;\;} - \frac{1-{\alpha_r}}{2} \l|a\r| \l\{\urdrL
- \urdrR[n-1] \r\} \bigg]  \cdot \urdrL\\
 &+  \bigg[ \frac{a}{2} \l\{\mbox{\;\;\:} \urdrL[n+1] - \urdrR[n] \r\} \\
 &\mbox{\;\;\;} - \frac{1-{\alpha_r}}{2} 
\l|a\r|
\l\{\urdrL[n+1] - \urdrR[n] \r\} \bigg]
 \cdot \urdrR 
\bigg)
 \bigg\} 
\end{split}
\end{equation}


Distributing the $\l(\frac{1}{J_n^{2r}}\r)$ term to convert the derivatives with respect to $\xi$
into derivatives with respect to $x$, and factoring out the $ \frac{a}{2}$ term,

\begin{equation}
\label{eq:termaphys}
\begin{split}
 \text{\textcircled{a}} = -\sum_{n=1}^{N}  \bigg\{
 \sum^m_{r=0} & c_r \frac{a}{2}
 \bigg(  
 \l[ {u}^{{ (r)}^2} \big|_{R,n} - {u}^{{ (r)}^2} \big|_{L,n} \r] \\
 %
 &- \bigg[ \l\{-\uudrL[n] + \uudrR[n-1] \r\} \\
 &\mbox{\;\;\;} - \frac{1-{\alpha_r}}{a} \l|a\r| \l\{\uudrL -
\uudrR[n-1] \r\} \bigg]
%
 \cdot \uudrL\\
 &+  \bigg[ \l\{\mbox{\;\;\:} \uudrL[n+1] - \uudrR[n] \r\} \\
 &\mbox{\;\;\;}- \frac{1-{\alpha_r}}{a} \l|a\r|
\l\{\uudrL[n+1] - \uudrR[n] \r\} \bigg]
 \cdot \uudrR 
\bigg)
 \bigg\} 
\end{split}
\end{equation}
%where the superscript $(r)$ now represents the $r^\text{th}$ derivative with respect to $x$.

Note that all terms in Eqn.~\eqref{eq:termaphys} are defined at element interfaces. More explicitly,
\begin{equation}
\label{eq:uudIdentities}
\begin{split}
\uudrL &= \uud_n(x_n) \\
\uudrR &= \uud_n(x_{n+1})
\end{split}
\end{equation}
where $\uud_n$ is the polynomial representing the solution in element $n$ and $x_n$ is the location
of the $n^\text{th}$ interface in physical coordinates.
Using the identities in Eqn.~\eqref{eq:uudIdentities}, Eqn.~\eqref{eq:termaphys} becomes
\begin{equation}
\label{eq:telSum0}
\begin{split}
 \text{\textcircled{a}} &= -\sum_{n=1}^{N}  \bigg\{
 \sum^m_{r=0} c_r \frac{a}{2}
 \bigg(  
 \l[ \l\{\uudrx{n}{n+1}\r\}^2  - \l\{\uudrx{n}{n}\r\}^2 \r] \\
 %
 &- \bigg[ \l\{-\uudrx{n}{n} + \uudrx{n-1}{n} \r\} \\
 &\mbox{\;\;\;}- \frac{1-{\alpha_r}}{a} \l|a\r|
\l\{\uudrx{n}{n} 
-
\uudrx{n-1}{n} \r\} \bigg]
%
 \cdot \uudrx{n}{n}\\
 &+  \bigg[ \l\{ \uudrx{n+1}{n+1} - \uudrx{n}{n+1} \r\} \\
 &\mbox{\;\;\;}- \frac{1-{\alpha_r}}{a} \l|a\r|
\l\{\uudrx{n+1}{n+1} - \uudrx{n}{n+1} \r\} \bigg]
 \cdot \uudrx{n}{n+1} 
\bigg)
 \bigg\} 
\end{split}
\end{equation}

Let us do the following substitutions to simplify algebraic manipulations
\begin{equation}
\label{eq:Bn}
\begin{split}
B_n = \sum^m_{r=0} c_r \frac{a}{2} \bigg( &\l[\uudrx{n-1}{n}\r]^2  \\
+ & \bigg[ \l\{\uudrx{n}{n} - \uudrx{n-1}{n} \r\} \\
&\mbox{\;\;\;}- \frac{1-{\alpha_r}}{a} \l|a\r| \l\{\uudrx{n}{n} 
- \uudrx{n-1}{n} \r\} \bigg] \cdot \uudrx{n-1}{n} \bigg)
\end{split}
\end{equation}

\begin{equation}
\label{eq:Dn}
\begin{split}
 D_n = \sum^m_{r=0} c_r \frac{a}{2} \bigg(- &\l[\uudrx{n}{n}\r]^2 \\
 - & \bigg[ \l\{-\uudrx{n}{n} + \uudrx{n-1}{n} \r\} \\
 &\mbox{\;\;\;}- \frac{1-{\alpha_r}}{a} \l|a\r|
\l\{\uudrx{n}{n} 
- \uudrx{n-1}{n} \r\} \bigg] \cdot \uudrx{n}{n} \bigg)
\end{split}
\end{equation}



We can then rewrite Eqn.~\eqref{eq:telSum0} as
\begin{equation}\label{eq:simple_summation}
  \text{\textcircled{a}} = -\sum_{n=1}^{N} \l\{ B_{n+1} + D_{n} \r\} 
\end{equation}

Let us manipulate \eqref{eq:simple_summation} to combine the two summations into one whose terms have the same unshifted index
\begin{equation}
\begin{split}
  \text{\textcircled{a}} &=  - \sum_{n=1}^{N} D_{n} -\sum_{n=1}^{N}  B_{n+1} \\
  \text{\textcircled{a}} &= - \sum_{n=1}^{N} D_{n} -\sum_{n=2}^{N+1}  B_{n}  \\
  \text{\textcircled{a}} &= - D_{1} - \sum_{n=2}^{N} D_{n}  - \sum_{n=2}^{N}  B_{n}  - B_{N+1} \\
  \text{\textcircled{a}} &= - D_{1} - \sum_{n=2}^{N} \l\{ D_{n}  +  B_{n} \r\}  - B_{N+1}
\end{split}
\end{equation}

Note that various terms in $B_n$ and $D_n$ cancel when summed (compare \eqref{eq:Bn} and \eqref{eq:Dn}). After doing such cancellations,
Eqn.~\eqref{eq:telSum0} becomes
\begin{equation}
 \text{\textcircled{a}} = -D_1 - \sum_{n=2}^{N} \sum^m_{r=0} c_r
\cdot \frac{1-\alpha_r}{2} |a| \l(\uudrx{n-1}{n} - \uudrx{n}{n} \r)^2 - B_{N+1} 
\end{equation}

The value of the solution and its derivatives at $x_1$ and $x_{N+1}$ are set by the desired boundary conditions. Both $D_1$ and $B_{N+1}$ depend exclusively on such pre-determined conditions:
% \begin{equation}
%  \uudrx{n}{n} = \uudrx{n-1}{n}
% \end{equation}
% As a result,
\begin{align}
 D_1 &= -\sum^m_{r=0} c_r\frac{a}{2} \l[ \uudrx{1}{1}\r]^2 \\
 B_{N+1} &= \sum^m_{r=0} c_r\frac{a}{2} \l[ \uudrx{N}{N+1}\r]^2
\end{align}

To not introduce/extract energy into/from the solution, let us set periodic boundary conditions in all derivatives,
\begin{equation}
 \uudrx{1}{1} = \uudrx{N}{N+1}
\end{equation}

Consequently, $D_1 + B_{N+1} = 0$, and Eqn.~\eqref{eq:telSum0} becomes simply
\begin{equation}
 \text{\textcircled{a}} = - \sum_{n=2}^{N} \sum^m_{r=0} c_r
\cdot \frac{1-\alpha_r}{2} |a| \l(\uudrx{n-1}{n} - \uudrx{n}{n} \r)^2
\end{equation}

Knowing that the following holds,
\begin{equation}
 \begin{split}
  c_r &\ge 0\\
  0 \le \alpha_r &\le 1\\
  \l(\uudrx{n-1}{n} - \uudrx{n}{n} \r)^2 &\ge 0  \mbox{ for } 2\le n \le N
 \end{split}
\end{equation}

we conclude that
\begin{equation}
 \text{\textcircled{a}} \le 0
\end{equation}


We have shown that the term \textcircled{a} in Eqn.~\eqref{eq:timenorm} is non-positive,
concluding this part of the proof.


\subsection{Part 2.}
We wish to find functions $\gL $ and $\gR $, for $i = 0,\ldots,m $,
which satisfy the boundary conditions in Eqn.~\eqref{eq:gconstraints} and the stability conditions of Eqn. \eqref{eq:bc_equal_0} (\textcircled{b}$=0$
and \textcircled{c}$=0$) for arbitrary
$c_r, r=1,\ldots,m$.

We start by finding $\gL$. Let us manipulate \textcircled{b} in Eqn. \eqref{eq:b_def} to find restrictions on $\gL$,

% done: 09/24/2014
\begin{equation}
\begin{split}\label{eq:b_cond}
 \text{\textcircled{b}} = \sum_{n=1}^{N} \l\{ 
 \sum^m_{r=0} \l[ c_r \l(
 \frac{1}{J_n^{2r}}\r) \sum^m_{i=0} \l( \IL \ioo \dD{\gL}{r} \hat{u}^{{ (r+1)}} d\xi \r) \r] \r\} &=0 \\
 = \sum_{n=1}^{N} \l\{ 
  \sum^m_{i=0} \l[ \IL \sum^m_{r=0} \l( c_r 
  \frac{1}{J_n^{2r}}  \ioo \dD{\gL}{r} \hat{u}^{{ (r+1)}} d\xi \r) \r] \r\} &=0
 \end{split}
\end{equation}

To satisfy Eqn. \eqref{eq:b_cond}, either
\begin{equation}
\label{eq:optionA}
\sum^m_{i=0}\l( \IL \ioo \dD{\gL}{r} \hat{u}^{{ (r+1)}}
d\xi \r) =0 
\end{equation}
or
\begin{equation}\label{eq:optionB}
\sum^m_{r=0} \l( c_r 
  \frac{1}{J_n^{2r}}  \ioo \dD{\gL}{r} \hat{u}^{{ (r+1)}} d\xi \r) =0
\end{equation}


We then have two options for finding $\gL$. Observing that the only term in the summand in \eqref{eq:optionB} that changes as the solution evolves is $\urd$ itself, we realize that a generic $\gL$ with special polynomial orthogonality properties would satisfy Eqn. \eqref{eq:b_cond}. As a result, we could pre-compute a non-changing $\gL$ to run a linearly stable scheme. On the other hand, the option given by \eqref{eq:optionA} obliges us to find
$\gL$ at every time-step because the scalar $\IL[i]$ changes with the solution. Therefore, we choose to find
$\gL$ using \eqref{eq:optionB}. %:
%\begin{equation}
%\label{eq:glcond}
% \sum^m_{r=0} c_r   \ioo \dD{\gL}{r} \hat{u}^{{\delta (r+1)}} d\xi \l(
%\frac{1}{J_n^{2r}}\r) =0 \text{ for each element $n$}
%\end{equation}


Rewriting $\urd$ as a sum of scaled Lagrange polynomials, condition \eqref{eq:optionB} becomes
\begin{equation}
 \label{eq:glcond1}  \sum^{P+1}_{p=1} \urd_p \sum^m_{r=0} \frac{c_r}{J^{2r}_n} \l(\ioo \dD{\gL}{r}
\dD{l_p}{r+1} d\xi \r) = 0
\end{equation}

This implies that
\begin{equation}\label{eq:glcond}
\sum^m_{r=0}\frac{c_r}{J^{2r}_n} \ioo \dD{\gL}{r} \dD{l_p}{r+1} d\xi = 0 
\end{equation}
for $p=1,\ldots, P+1$, (recall $P$ is the order of the polynomial used to represent $\urd$).
Let us expand the Lagrange polynomials $l_p$ and the correction functions $\gL$ into monomials,
\begin{equation}
\begin{split}
 l_p&=\sum^P_{j=0} \zeta_{pj} \xi^j\\
 \gL &= \sum^S_{k=0} \theta_{ik} \xi^k 
 \end{split}
\end{equation}
Note that $\zeta_{pj}$ are known, unchanging scalars while $\theta_{ik}$ are the unknowns we are trying to find. $S$ is the desired polynomial order of the correction function.

Eqn.~\eqref{eq:glcond} becomes
 \begin{equation*}
\sum^m_{r=0} \frac{c_r}{J^{2r}_n} \ioo 
\l( \sum^S_{k=r} \theta_{ik} \frac{k!}{(k-r)!} 
\xi^{k-r}  \r)
\l(
\sum^P_{j=r+1}\zeta_{pj}\frac{j!}{(j-r-1)!}  \xi^{j-r-1} \r)
d\xi =0
\end{equation*}

After some algebraic manipulation, we arrive at the conditions that each $\gL$ must satisfy so the \gls{cmfr} scheme maintains linear stability:
\begin{equation}
 \label{eq:gcond}
\sum^m_{r=0} \frac{c_r}{J^{2r}_n}  
\l[ \sum^S_{k=r} \theta_{ik} \frac{k!}{(k-r)!} 
\l(
\sum^P_{j=r+1}\zeta_{pj}\frac{j!}{(j-r-1)!}  \ioo \xi^{j+k-2r-1} d\xi\r)\r]
 =0
\end{equation}

Recall that $i$ in $\theta_{ik}$ indexes the correction function corresponding to each specific derivative in which continuity is desired. $p$ indexes each solution point in the reference domain.

With Eqn.~\eqref{eq:gcond} and the constraints given in Eqn.~\eqref{eq:gconstraints} we can
construct a system of equations to solve for $\gL$ in each element. More specifically, if we
would like to ensure $s$ different flux continuities between elements, the boundary constraints in
Eqn.~\eqref{eq:gconstraints} produce $2s$ equations, and the conditions in
Eqn.~\eqref{eq:gcond} produce $P$ equations (not necessarily all independent), where $P$ is the order of the solution
representation $\urd$. %Note, however, that if the flux continuity is desired in the $P+1^\text{th}$, or higher, derivative, the flux can be of order $2s+P-1+(-2)$ or higher, as the boundary conditions related to the $P+1^\text{th}$ derivative are satisfied automatically due to the fact that $\hat{u}^{\delta(P+1)}=0$ in every element.

We can recover the \gls{esfr} scheme by
ensuring continuity in the $0^\text{th}$ and $P+1^\text{th}$ derivatives. These two desired
flux continuities ($s=2$) force the flux to be of order, at least, $P+1$. This is the order of the correction functions suggested by Vincent~\cite{vincent2011new}. The \gls{esfr} scheme is one of the cases in which Eqn. \eqref{eq:glcond} produces $P-1$ independent equations instead of $P$. This can be seen by the fact that the $P+1^\text{th}$ and  $P+2^\text{th}$ derivatives of a Lagrange polynomial $l_p$ of order $P$ is zero.

Once we find $\gL$, we find $\gR$ automatically by setting 
\begin{equation}\label{eq:gr_gl_cond}
 \gR = (-1)^i\gL(-\xi)
\end{equation}
%in Eqn. \eqref{eq:c_def}.