\section{Flux Reconstruction Method}
\label{sec:frmethod}
What follows is a brief overview of the \gls{fr} framework with an emphasis on how the \gls{lfs} filtering technique can be seamlessly implemented. We start the discussion with the solution of the advection equation in one dimension using the \gls{fr} approach to illustrate the method. We then proceed to briefly explain how conservation equations can be solved in multiple dimensions. The \gls{ns} equations are a set of coupled conservation equations in multiple dimensions, so the extension of the \gls{fr} methodology to them is straightforward. The detailed description of the algorithm used in \gls{hf} is given by Castonguay et al.~\cite{castonguay2011}.

\subsection{Solution of the Advection Equation in One Dimension using the FR Approach}

Consider the one-dimensional conservation law
\begin{equation}
\label{eq:cons}
\frac{\partial u}{\partial t} + \frac{\partial f}{\partial x} = 0
\end{equation}

in domain $\Omega$, where $x$ is the spatial coordinate, $t$ is time, $u$ --the \emph{solution}-- is a scalar function of $x$ and $t$, and $f$ --the \emph{flux}-- is a scalar function of $u$. Note that by letting $f = f(u,\frac{\partial u}{\partial x})$, Equation~\eqref{eq:cons} becomes a model of the Navier-Stokes equations.

Let us partition the domain $\Omega = [x_1,x_{N+1})$ into $N$ non-overlapping elements with 
interfaces at $x_1<x_2<...<x_{N+1}$. Then,
\begin{equation}
\Omega = \bigcup^N_{n=1} \Omega_n
\end{equation}
and $\Omega_n = [x_n,x_{n+1})$ for $n = 1,...,N$. To simplify the implementation, let us map each of the physical elements $\Omega_n$ to a standard element $\Omega_s=[-1,1)$ with the function $\Theta_n(\xi)$, where
\begin{equation}
x = \Theta_n(\xi) = \l( \frac{1-\xi}{2} \r) x_n + \l(\frac{1+\xi}{2}\r) x_{n+1} 
\end{equation}

With this mapping, the evolution of $u$ within each $\Omega_n$ can be determined with the following 
transformed conservation equation
\begin{equation}
\frac{\partial \hat{u}}{\partial t} + \frac{1}{J_n}\frac{\partial \hat{f}}{\partial \xi} = 0,
\end{equation}
where
\begin{equation}
\hat{u} = u(\Theta_n(\xi),t) \text{ in } \Omega_n,
\end{equation}
\begin{equation}
\hat{f} = f(\Theta_n(\xi),t) \text{ in } \Omega_n,
\end{equation}
\begin{equation}
J_n = \frac{\partial x}{\partial \xi} \bigg|_{\Omega_n}.
\end{equation}

Now, we introduce polynomials of degree $p$, $\hat{u}^\delta$ and $\hat{f}^\delta$, to approximate the exact values $\hat{u},\hat{f}$, respectively. We can write these polynomials as
\begin{equation}
\hat{u}^\delta = \sum_{i=1}^{N_s} \hat{u}_i^\delta l_i(\xi),
\end{equation}
\begin{equation}
\hat{f}^\delta = \sum_{i=1}^{N_s} \hat{f}_i^\delta l_i(\xi),
\end{equation}
where $N_s$ is the number of solution points, $\hat{u}_i^\delta$ is the current value of the 
solution approximation function at the $i^\text{th}$ \emph{solution point} (or \emph{internal point}) in the reference element, 
$\hat{f}_i^\delta$ is the current value of the flux approximation function at the $i^\text{th}$ 
\emph{flux point} (or \emph{boundary point}) in the reference element, $l_i$ is the Lagrange polynomial equal to $1$ at the 
$i^\text{th}$ solution point and $0$ at the others, and $\delta$ denotes that the function is an 
approximation.

Note that the piecewise polynomials might not be continuous (or $C^0$) across the interfaces. In the 
Flux Reconstruction approach, the flux used in the time advancement of the solution is made $C^0$ 
by introducing flux correction functions.

This can be achieved by finding interface solution values at each element boundary and then correcting the 
solution. Let $\hat{f}_L^{\delta I}$ and $\hat{f}_R^{\delta I}$ be the interface flux values at left and right 
boundaries of some element, respectively. $\hat{f}_L^{\delta I}$ and $\hat{f}_R^{\delta I}$ can be found with a Riemann solver for \gls{dg} methods\cite{hesthaven2007nodal}. Then, select solution correction functions $h_L$ and 
$h_R$ such 
that
\begin{equation}\label{eq:condition}
h_L(-1) = 1 \;,\; h_L(1) = 0,
\end{equation}
\begin{equation}
h_R(-1) = 0 \;,\; h_R(1) = 1,
\end{equation}
and let
\begin{equation}
\hat{f}^C = \hat{f}^\delta + (\hat{f}^{\delta I}_L - \hat{f}^\delta_L) h_L + (\hat{f}^{\delta I}_R 
- \hat{f}^\delta_R) h_R,
\end{equation}
where superscript $C$ denotes the function is corrected, and $\hat{f}^\delta_L$, $\hat{f}^\delta_R$ 
represent the flux approximation evaluated at the left and right boundaries.

The solution can then be advanced at each solution point $i$ in element $n$. In semi-discrete form, this is
\begin{equation}\label{eq:semidiscrete}
\frac{d \hat{u}_i^\delta}{d t} = - \frac{1}{J_n}\frac{\partial \hat{f}^C}{\partial \xi}(\xi_i).
\end{equation}

The \gls{fr} scheme can be made provably stable for the linear advection-diffusion equation by selecting special types of correction functions~\cite{castonguay2013energy}. In general, these correction functions are polynomials of degree $p+1$ so both sides in Equation~\eqref{eq:semidiscrete} are quantities related to polynomials of order $p$ --for consistency~\cite{huynh2007flux}.

Vincent et al.~\cite{vincent2011new} have shown that in the case of the 1-dimensional, linear advection equation, the \gls{fr} approach can be proven to be stable for a specific family of correction functions parameterized by a scalar called $c$. In addition, they showed that by selecting specific values of $c$ it is possible to recover a particular nodal \gls{dg} and \gls{sd} methods plus a \gls{fr} scheme that was previously found to be stable by Huynh\cite{huynh2007flux}.

\subsection{Extension to Multiple Dimensions}

Extension of \gls{fr} to multiple dimensions requires formulating multi-dimensional interpolation functions and correction functions that satisfy boundary conditions equivalent to those in Equation~\eqref{eq:condition} for each type of element.

Interpolation bases for quadrilaterals and hexahedra can be obtained via tensor products of the 1-dimensional interpolation basis. In \gls{hf}, the solution in hexahedra is discretized in the following way
\begin{equation}
{\hat{u}}^\delta(\xi,\eta,\zeta) = \sum_{i=1}^{p+1} \sum_{j=1}^{p+1} \sum_{k=1}^{p+1}
{\hat{u}}^\delta_{i,j,k} l_i(\xi) l_j(\eta) l_k(\zeta),
\end{equation}
where $i$, $j$, $k$ index the solution points along the $\xi, \eta, \zeta$ directions, respectively. The flux is discretized similarly.

The interpolation basis for triangles and tetrahedra are described in detail by Hesthaven and Warburton~\cite{hesthaven2007nodal}. Figure \ref{fig:tri_points} shows a possible configuration of internal and boundary points. The extension of interpolation polynomials to prisms is obtained via tensor products of the 1-dimensional basis with the triangular basis~\cite{castonguay2011}. 

The most general polynomial discretization of a $k$-dimensional solution scalar field and flux vector field in an arbitrary reference element is
\begin{align}
\label{eqn:general_u}
\hat{u}^\delta({\pmb \xi}) &= \sum_{i=1}^{N_s} \hat{u}_i^\delta {\phi}_i({\pmb \xi}),\\
\hat{\bf f}^\delta({\pmb \xi}) &= \sum_{i=1}^{N_s} \hat{\bf f}_i^\delta {\phi}_i({\pmb \xi}),
\end{align}
where $N_s$ is the number of solution points in an element, ${\phi}_i({\pmb \xi})$ is a polynomial basis function associated with solution point $i$ constructed such that ${\phi}_i({\pmb \xi}_j) = \delta_{ij}$ and $i,j = 1,\dots,N_s$.

By letting $\vec{u} = <\hat{u}_1^\delta, \hat{u}_2^\delta,\dots,\hat{u}_{N_s}^\delta>^T $, $\vec{\bf f} = <\hat{\bf f}_1^\delta, \hat{\bf f}_2^\delta,\dots,\hat{\bf f}_{N_s}^\delta>^T $, and $\vec{\phi} = <\phi_1,\phi_2,\dots,\phi_{N_s}>^T$ the discretization can be written more concisely as
\begin{equation}
\begin{split}
\hat{u}^\delta({\pmb \xi}) &= \vec{u}^T \cdot \vec{\phi}({\pmb \xi}) =  \vec{\phi}({\pmb \xi})^T\cdot \vec{u},\\
\hat{\bf f}^\delta({\pmb \xi}) &= \vec{\bf f}^T \cdot \vec{\phi}({\pmb \xi}) = \vec{\phi}({\pmb \xi})^T \cdot \vec{\bf f}.
\end{split}
\end{equation}

Note that we are using the boldface and arrow notation to denote vectors. Boldface vectors have a number of entries equal to the number of dimensions of the problem domain. Arrow vectors are vectors of general dimensions.

In the general \gls{fr} approach, the boundary conditions for the correction functions in multiple dimensions can be formulated as
\begin{equation}\label{eq:genConstraint}
{\bf{h}}_i( {\pmb \xi}_j)\cdot {\bf{n}}_j = \delta_{ij},
\end{equation}
where ${\bf{h}}_i$ is the correction vector function associated with interface point $i$, ${\pmb \xi}_j$ is the location vector of the $j^\text{th}$ interface point, and ${\bf{n}}_j$ is the outward unit normal at interface point $j$. Interface (or boundary) points are located on the boundary of an element.

One of the challenges in the \gls{fr} approach is finding correction functions that not only satisfy Equation~\eqref{eq:genConstraint} but also guarantee stability in the linear advection-diffusion case. Correction functions that guarantee such stability exist for 1-dimensional segments\cite{vincent2011new}, triangles\cite{castonguay2012new,williams2013tri}, and tetrahedra\cite{williams2013tet}. FR schemes with these correction functions comprise the ESFR family of schemes.

The update step at solution point $i$ and element $n$ in the \gls{fr} approach for the multidimensional advection equation $\frac{\partial u}{\partial t} + \nabla \cdot \vec{f} = 0$ becomes
\begin{equation}
\label{eqn:discrete}
\begin{split}
\frac{d {u}^\delta_i}{d t} &= - \frac{1}{\text{det}(\tilde{J_n})} \nabla \cdot {\bf f}_i^C({\pmb \xi}_i) = - \frac{1}{\text{det}(\tilde{J_n})} \nabla \cdot \l({\bf f}_i^\delta({\pmb \xi}_i) + \sum^{N_f}_{j = 1} \l[\l( \hat{\bf f}^{\delta I}_j - \hat{\bf f}^b_j \r)\cdot {\bf n}_j \r]\vec{\bf h}_j ({\pmb \xi}_i) \r)\\
&=- \frac{1}{\text{det}(\tilde{J_n})} \l( \nabla \cdot {\bf f}_i^\delta({\pmb \xi}_i) + \sum^{N_f}_{j = 1} \l[\l( \hat{\bf f}^{\delta I}_j - \hat{\bf f}^b_j\r)\cdot {\bf n}_j \r]\nabla \cdot\vec{\bf h}_j ({\pmb \xi}_i)\r)\\
&=- \frac{1}{\text{det}(\tilde{J_n})} \l( \vec{\nabla \phi}({\pmb \xi}_i)^T  \cdot  \vec{\bf f} + \sum^{N_f}_{j = 1} \l[\l( \hat{\bf f}^{\delta I}_j - \hat{\bf f}^b_j\r)\cdot {\bf n}_j \r]\nabla \cdot\vec{\bf h}_j ({\pmb \xi}_i)\r),\\
\end{split}
\end{equation}
where $N_f$ is the number of interface points, $\tilde{J_n}$ is the Jacobian matrix, $\vec{\nabla \phi}({\pmb \xi}_i)$ is the vector of gradients of each $\phi_i$ function evaluated at ${\pmb \xi}_i$, $\vec{\bf f}^\delta$ is a vector of flux vectors, and $\hat{\bf f}^b_j$ is the flux vector at the $j^{\text{th}}$ interface point (obtained via extrapolation).

Note that it is possible to evaluate each of the terms in Equation \eqref{eqn:discrete} for all $i = 1,\dots,N_s$ with a series of matrix-vector multiplications.


