%Purpose and scope; methods; key results; contributions to the state of the art; references
\section{Purpose}
       Jet impingement is used in multiple industrial and engineering applications. It is most commonly used to entrain particles on a surface (as in the printing industry) and to extract heat from a surface. In the aerospace industry, jets impinging on concave surfaces can be used to cool down high-pressure turbine blades\cite{han2001}. In this application, jet arrays impinge on the inner edges of blades. Given the complexity of the applications, accurate numerical simulations are an essential tool in the design of efficient impinging jets.

Jefferson et al. \cite{jefferson2010} compared the results of different numerical methods to experiments peformed by Lee et al. \cite{lee1997} of jets impinging on concave surfaces. Jefferson et al. acknowledge that the numerical results obtained with LES were very sensitive to the boundary conditions. Additionally, they found that the numerical scheme they used for LES was too dissipative and, in order to get more accuracy, needed to use a wall-resolved grid.

In the current study we use a high-order numerical scheme to limit the numerical dissipation and explore further the effects of the geometric representation of the boundary on the LES simulations.



