\section{Introduction}

The flux reconstruction (FR) approach provides a common framework for popular higher order methods. Huynh \cite{Huynh07} has shown that the FR approach encompasses both collocation-based nodal discontinuous galerkin (DG) schemes of the type described by Hesthaven and Warburton\cite{Hesthaven08}, and (in linear problems) the spectral difference (SD) method, which was originally proposed by Kopriva and Kolias \cite{Kopriva96}, and later generalized by Liu, Vinokular and Wang \cite{Liu06}. Recently a new range of FR schemes (referred to as Energy Stable Flux Reconstruction (ESFR) schemes) have been identified. These scheme are unique in that they are linearly stable for all orders of accuracy in a norm of Sobolev type. Utilizing a FR formulation, Jameson proved that (for 1D linear advection) a particular SD method is stable for all orders of accuracy in a norm of Sobolev type \cite{Jameson10}. Recently, this result has been extended by Vincent, Castonguay and Jameson \cite{Vincent10}, who identified an infinite range of linearly stable FR schemes. These schemes, henceforth referred to as ESFR schemes, are parameterized by a single scalar $c$. The identification of such schemes offers significant insight into why certain FR schemes are stable, whereas others are not. Also from a practical standpoint the ESFR formulation offers a simple prescription for implementing an infinite range of efficient and linearly stable high-order methods. 

\vspace{0.1 in}

\noindent There is a distinct set of challenges associated with treating the viscous terms within the context of ESFR and other high order schemes. The discontinuous nature of the solution representation is fundamentally in conflict with the continuous nature of a diffusion process.  A continuous viscous flux is required at the boundaries between cells to couple the viscous phenomenon in adjacent cells, providing consistency and stability. In general, forming this continuous viscous flux is non-trivial because of the two-fold discontinuity at the cell periphery. In the most general case, the viscous flux depends on two quantities (the solution and its gradient) which are discontinuous at the cell boundary. An algorithm (usually based on the details of the high order scheme) is required to determine acceptable common values for both the solution and its gradient at the boundary.  

\vspace{0.1 in}

\noindent Current approaches for forming these common values can be credited to researchers who applied them within in the context of DG methodologies. The standard procedure is to take a weighted average of the solution on either side of the cell interface. When the solutions on either side of the interface are equally weighted, one recovers the arithmetic average. The arithmetic average approach is credited to Bassi and Rebay (BR) \cite{Bassi97}. It is perhaps the most intuitive, as there is no physical reason to favor solution information from one side of the interface relative to the other. A less intuitive approach (originally due to Cockburn and Shu \cite{Cockburn98}) utilizes the one-sided average. Physical consistency is maintained by taking turns, favoring one cell and then the other. In 1D, if the common solution is biased towards the the left cell, the common gradient is computed so that it is biased towards the right cell. For linear problems, this approach has proven to yield an additional order of accuracy above that of Bassi and Rebay \cite{Huynh09}. However, for non-linear problems both approaches produce the same order of accuracy (k+1).  The one-sided approach is commonly referred to as LDG. It is referred to as CDG when reformulated in a compact construction, using only the cell's nearest neighbors \cite{Peraire08}. Similarly, the BR approach has appeared in non-compact (BR1)\cite{Bassi97} and compact (BR2) \cite{Bassi00} forms. 

\vspace{0.1 in}

\noindent Within the differential FR context, forming the common solution (of BR2 or CDG) is straightforward. It is formed using the weighted average (as discussed above). However, an approach for forming the common solution gradient is non-trivial. The BR2 and CDG approaches for constructing the common gradient depend on the quadrature-based aspect of traditional DG methods. In particular, the BR2 approach involves the use of quadrature-based lifting operators \cite{Bassi97}. These operators are used to correct the solution gradient at the element boundary based on the jump in the gradient. FR schemes are fundamentally constructed in a differential, quadrature-free formulation. Lifting terms are not naturally present, and thus the development of viscous terms for FR schemes remains an open area of research. Previous research in this area has been limited to linear model problems on regular grids (Huynh \cite{Huynh09}), or has required the formulation of additional quadrature based terms (Gao \cite{Gao09}). The purpose of this paper is to extend FR schemes to treat viscous terms in a quadrature-free framework. This effort will culminate in the application of FR schemes to viscous, non-linear problems around complex geometries. This will entail an extension of the viscous FR formulation to the Navier-Stokes equations in 2D, as well as an extension to unstructured grids with triangular elements.

\vspace{0.1 in}

\noindent This paper is structured as follows. First, a review of FR is presented, followed by a brief introduction to ESFR schemes. Next, several FR-based approaches for treating the viscous terms are presented. In this section, important concerns and open questions for each approach are presented. These questions are answered through a series of numerical experiments on linear model problems (in 1D) and on the non-linear Navier Stokes equations (in 2D). Finally, the paper concludes with a summary of the results and a discussion of avenues for future research.
