\section{Introduction}
Low-order methods are ubiquitous in industry and academia. Regardless of the mesh quality and flow conditions, commercial \gls{cfd} packages output an answer. It is up to the informed user to decide if such answer is believable or accurate to her satisfaction. This is not so true of high-order methods: they are still sensitive to starting conditions, mesh quality, and the non-linearity of the flow, i.e. how high \gls{re} is. If any of these parameters is not chosen well, the simulation will halt prematurely and no result, not even a rough estimate, will be provided. Certainly, this is not acceptable in an industrial setting.

The \gls{lfs} filters being proposed in this chapter are an attempt at tackling the low robustness of high-order methods from the perspective of flow physics, rather than the classical frameworks of polynomial order reduction, artificial viscosity, or limiting. Very promising results have been shown by Asthana and the author~\cite{asthana2014} for 1D non-linear advection-diffusion problems and 2D inviscid high-\gls{ma} flows with quadrilateral, unstructured, coarse grids, and polynomial discretizations of up to order $8$. The present paper extends their formulation to arbitrary elements in arbitrary dimensions and shows results in high-\gls{re} flows without turbulence modeling.

It is well known that turbulent flows --which tend to be high-\gls{re}--, exhibit an energy cascade: the energy from large scales is transferred to smaller scales due to natural dissipation. In very crude terms, large vortices become ever smaller vortices until they reach dimensions proportional to the Kolmogorov length-scale~\cite{kolmogorov1962refinement}. This phenomenon is well captured by the \gls{ns} equations. Hence, a good \gls{ns} solver would see large scales become ever smaller. In general, low-order methods in grids not created for \gls{dns} introduce enough numerical dissipation that the ever shrinking scales are dissipated before they become aliased (or under-sampled).

We postulate that it is precisely this very natural energy cascade which is de-stabilizing high-order numerical methods. Because high-order methods introduce little numerical dissipation, the ever shrinking scales may not be dissipated before they become aliased: they re-appear as larger scales that, naturally, become smaller later on. This vicious cycle introduces non-physical energy into the flow until the simulation is de-stabilized. Thus, removing the small scales before they become aliased would, in theory, stabilize the solution.

\gls{lfs} filters target scales relative to the element size, as the filtering operation happens in the reference element. The smaller the element, the smaller the scale being filtered. In addition, the filters help satisfy boundary conditions. The  results presented here show that the \gls{lfs} filters can  stabilize not only high-$\Re$ flows but also moderate-$\Ma$ and low-$\Ma$ flows in coarse grids, which opens the door to using the filters as pre-conditioners or in multigrid cycles. All simulations being presented ran from start to finish without intervention.

Many stabilization schemes created for high-order methods have focused on shock-capturing. A concise review of the shock-capturing literature can be seen in Section 3.4 in~\cite{vincent2011facilitating}. An \gls{av}-based shock capturing approach that has gained popularity because of its ease of implementation and increase of robustness was suggested by Persson and Peraire~\cite{persson2006sub}. A similar approach for high-$\Re$ flows with turbulence modeling has been proposed by Nguyen et al.~\cite{nguyen2007rans}. Lodato~\cite{lodato2014structural} has used filtering in the formulation of \gls{sgs} models for \gls{les} with high-order \gls{sd} schemes. His work inspired the formulation of the filters presented here. A stabilization strategy based on optimization was suggested by Guba et al.~\cite{guba2014optimization} shows great promise. A limiter-based stabilization strategy easily implemented in \gls{dg}-type methods was proposed by Kuzmin~\cite{kuzmin2004high} and Lv et al.~\cite{lv2015entropy}.

The main reason we decided to find a stabilization strategy that could be posed as a matrix multiplication and requires a very local stencil arises from the fact that \gls{hf} performs best on \gls{gpu}s. \gls{gpu}s require a low-communication, highly-parallel implementation with organized memory accesses and homogeneous computations. Implementation of \gls{av} would have required elements adjacent to each other to share information about the \gls{av} that each requires, leading to inevitable additional inter-element communication.

 Section \ref{sec:frmethod} presented a general description of the \gls{fr} method. Section \ref{sec:extension_multiple_dimensions} showed how this scheme relies on matrix multiplications and, hence, a filter that costs two small matrix multiplications per element is ideal for \gls{fr}. Section \ref{sec:lfsfilter} describes the properties being sought in the \gls{lfs} filters and the mechanics of their implementation. Section \ref{sec:visualization} provides visualizations of the filters and their effects on a polynomial solution. Section \ref{sec:results} presents the results of 2D simulations, in unstructured coarse grids, of flows where \gls{re}$= 1e6$. This \gls{re} number was selected because of the availability of experimental data for the case of the circular cylinder and \gls{hf} had not been run at these \gls{re} numbers. In addition, Section \ref{sec:filteringEffects} shows results that isolate the effects of filtering from the effects of coarsening a grid or changing the spatial order of accuracy to demonstrate that \gls{lfs} filters preserve element-wise spectral properties.

