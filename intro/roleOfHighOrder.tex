\section{The Role of High-Order Methods in the Future of Industrial Simulations}

Over the last 20 years, much fundamental work has been done in developing high-order numerical methods for Computational Fluid Dynamics. Moreover, the need to improve and simplify these methods has attracted the interest of the applied mathematics and the engineering communities. Now, these methods are beginning to prove themselves sufficiently robust, accurate, and efficient for use in real-world applications.

However, low-order numerical methods are still the standard in the aeronautical industry. There has been a sustained scientific and economical investment to develop this successful and robust technology for a long time. Currently, an industry-standard, second-order finite volume computational tool performs adequately well in a broad range of aeronautical engineering applications. For that reason, the introduction of new, high-order numerical schemes in the aeronautical industry is challenging, particularly in areas where the low-order numerical methods already provide the required robustness and accuracy (keeping in mind the limitations of current turbulence model technology).

Thanks to new and emerging aircraft roles (very small or large concepts, very high or low altitude, quiet vehicles, low fuel consumption vehicles, etc.), revolutionary aircraft design concepts will appear in the near future, and the need for high-fidelity simulation techniques to predict their performance is growing rapidly. Undoubtedly, high-order numerical methods are starting to find their place in the aeronautical industry. 

Unsteady simulations, flapping wings, wake capturing, noise prediction, and \gls{les} are just a few examples of computations that could benefit from high-order numerical methods. In particular, high-order methods have a significant edge in applications that require accurate resolution of the smallest scales of the flow. Such situations include the generation and propagation of acoustic noise from an airframe, or at the limits of the flight envelope where unsteady, vortex-dominated flows have a significant effect on aircraft performance. Utilizing a high-order representation enables smaller scales to be resolved with a greater degree of accuracy than standard second-order methods. Furthermore, high-order methods are inherently less dissipative, resulting in less unwanted interference with the correct development of the turbulent energy cascade. This factor makes the combination of high-order numerics with \gls{les} modeling very powerful, with the potential to significantly improve upon the accuracy and computational cost of the standard approach of \gls{les} with second-order methods. The amount of computing effort to achieve a small error tolerance can also be much smaller with high-order than second-order methods. Even real time simulations (one second of computational time, one second of real flight), could benefit from high-order algorithms that feature more intensive computation within each mesh element (ideal for vector machines and new computational platforms like GPUs, FPGAs, coprocessors, etc).

However, before claiming the future success of high-order numerical methods in industry, two main difficulties should be overcome: a) high-order numerical schemes must be as robust as state-of-the-art low-order numerical methods, b) the existing level of verification and validation in high-order CFD codes should be similar to the typical level of their low-order counterparts.