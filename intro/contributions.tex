\section{Contributions}

The vision of bringing a powerful simulation tool like \gls{hf} to the engineer's toolbox prompted me to focus on the areas that are essential in any such tool: validation and verification efforts, support to the open source community of researchers who want to use \gls{hf}, re-assessment of and contribution to the stability properties of \gls{fr} --\gls{hf}'s underlying numerical method--, and enabling \gls{hf} to provide a result even in the most extreme of simulation cases.
\subsection{Maintenance and Validation and Verification of \gls{hf}}
After exposure to SD++, \gls{hf}'s starting point, by creating its sparse matrix-vector multiplication routine~\cite{castonguay2011} in \gls{gpu}s, it became clear that anyone who has just started working on the source code needs guidance and assurances about its correctness. The \gls{vv} efforts undertaken by the \gls{acl}~\cite{lopez2014verification} provided the community assurances about its correctness. The crucial task of documenting \gls{hf}, creating tutorials, providing direct support to researchers, and eliminating bugs that are found by collaborators is an ongoing task, as can be seen in \gls{hf}'s repository: \url{github.com/HiFiLES/HiFiLES-solver}.

\subsection{Tackling Stability via Numerics: Creation of Provably Linearly Stable \gls{cmfr} Schemes}
When confronting the problem of the high propensity of \gls{hf} to Nan when a single element was not grided properly, or the flow non-linearities were even slightly too high for the grid, going back to the basics yielded interesting results. The \gls{cmfr} schemes~\cite{lopez2015cmfr} are provably stable for linear problems, very much the same way \gls{fr} is, but provide a range of selections of numerical schemes with varying dissipation and dispersion properties.

The academic endeavor of discovering these schemes provided valuable insights into the behavior and potential extensions of high-order methods. However, developing \gls{cmfr} schemes further would have detracted focus from the end goal of making \gls{hf} a robust tool.

\subsection{Tackling Stability via Filtering: Creation of Generalized \gls{lfs} Filters}
It was encouraging to see the results obtained by Asthana et. al~\cite{asthana2014} regarding stabilization of extreme high order computations in 1D and 2D tensor product elements using \gls{lfs}. The idea of filtering via truncated, element-wise convolutions of the solution with a kernel seemed to be too simple to work. Absolute accuracy in non-linear problems did increase with higher order approximations.

The extension to simplex 2D and 3D elements seemed straightforward. The extreme stabilizing properties of the filters were shown in~\cite{lopez2015stabilization}. The main difference between Asthana's formulation of the \gls{lfs} filters and the \gls{lfs} filters presented in this dissertation is that I allow the basis functions outside of the element being filtered to have any form (in fact, their form need not be defined), Asthana requires the basis functions to extend to infinity when computing the convolution.

My work regarding this form of stabilization has brought \gls{hf} closer to performing reliably in realistic industrial scenarios: complex, unstructured geometries with under-resolved grids being used to solve highly non-linear (high-\gls{re}, high-\gls{ma}) flows.





