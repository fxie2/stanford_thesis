\section{Motivation}
\newacronym{gpu}{GPU}{Graphical Processing Unit}
The SD++ code originally designed and developed by Peter Vincent, Patrice Castonguay, and David Williams \cite{castonguay2011} marked the culmination of multiple years of research at the \gls{acl}. It solved the Navier-Stokes and Euler equations in general unstructured grids in 2D and 3D. The original creators demonstrated its capabilities and excellent scalability in GPUs. Its stability in linear problems was guaranteed for triangular \cite{williams2013tri} and tetrahedral \cite{williams2013tet} elements with constant Jacobians, and its non-linear stability properties were understood to a practical extent \cite{jameson2012non}: as long as the exact flux of a hyperbolic equation being solved with the \gls{fr} scheme is not projected \emph{exactly} onto the polynomial space of the flux, aliasing instabilities will \emph{invariably} arise.

At the beginning of my graduate studies, I was extremely impressed by the capabilities of SD++, its clean code base, the high performance it achieved, and its general applicability beyond Aerospace applications. The members of the \gls{acl} agreed that a code with such capabilities could not be allowed to wither into oblivion. We set ourselves the goal, inspired by the success of fellow graduate students in the SU2 \cite{palacios2013stanford} team and guided by their lead developer -- Francisco Palacios--, to bring the capabilities of SD++ to a level where industrial applications (complex geometries, imperfect grids, high \gls{re}, high \gls{ma}) are feasible.

We concluded that releasing the code open source would allow us to more easily identify what capabilities are most useful and speed up development with the help of outside researchers. After adding additional \gls{les} \gls{sgs} models, local time-stepping, and artificial dissipation for shock-capturing, the \gls{acl} released the developer's edition of the code \gls{hf} on GitHub at \url{github.com/HiFiLES/HiFiLES-solver} with an official presence in the Stanford University servers at \url{hifiles.stanford.edu}.

Prof. Jameson's conclusion regarding aliasing in the solution of non-linear equations with \gls{fr} limited \gls{hf}'s applicability in the following ways:
\begin{itemize}
\item High gradients in the solution (shocks, fast moving fluid over a boundary) would lead to instabilities
\item Only medium to low \gls{re} flows could be simulated
\item Even in smooth problems, coarse meshing of areas far away form the regions of interest could lead to an unpredictable halt in the computations
\item Sharp corners in the geometry could stop the simulation
\item The time-step in mildly non-linear problems needed to be lower than expected to deal with the artificial stiffness introduced by aliased solutions
\end{itemize}

The thrill of getting a code as powerful and modern as \gls{hf} to produce an answer in all possible scenarios, very much how ANSYS Fluent and CFX can, was a strong drive to tackle the issues of stability.

The first idea to tackle de-stabilizing aliasing came while perusing the derivation of the \gls{esfr} schemes~\cite{vincent2011new} and the search for their optimal dispersion and dissipation properties~\cite{asthana2014high}. In Vincent's derivation, the correction functions are parameterized with a constant and are created so the corrected fluxes are globally continuous. The proof would work similarly if the correction functions were globally continuous in arbitrarily many derivatives, and thus had arbitrarily many parameterizing constants. This observation lead to the development of the \gls{cmfr} schemes~\cite{lopez2015cmfr}. Suddenly, a linearly provably-stable numerical scheme could have its dispersion and dissipation properties tuned as freely as necessary.

Approaching the stability issues from the numerics had the potential to yield good results, but a more immediate approach was needed to stabilize \gls{hf} in a general way. It was not clear \gls{cmfr} schemes could be readily applied to triangles and tetrahedral elements. Explicit filtering, as performed by Visbal et al.~\cite{visbal2003implicit}, seemed like an attractive prospect. White and Visbal use high-order Pad\'e differentiation and low-pass spatial filtering procedures in implicit \gls{les} to solve extremely complex flow scenarios like turbulence-shock interactions~\cite{white2015investigation}. The creation of the \gls{lfs} filters~\cite{asthana2014} and their extension to general, multi-dimensional elements~\cite{lopez2015stabilization} are the first attempt to bring the capabilities developed by Visbal to high-order codes for unstructured grids.
