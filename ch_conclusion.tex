% !TEX root = ./thesis.tex
\chapter{Conclusion}

Spurring the adoption of high-order methods in industry is a process. Not only is it necessary to show the advantages of these numerical schemes in regards to parallelizability, potential for scalability, and accuracy per degree of freedom, but also demonstrate that they can be as \emph{robust} and \emph{usable} as the current industrial tools.

Through the release, maintenance, and support of \gls{hf}, an open-source high-order code, the \gls{acl} aims to increase \emph{usability}. Feedback from fellow scientists and engineers can only help improve the learning curve needed to use high-order methods for practical purposes. This dissertation has shown validation and verification cases performed in \gls{hf} to demonstrate its versatility and provide a guide for researchers interested in performing similar computations.

The development of the \gls{cmfr} schemes provides a path to tuning dissipation and dispersion properties as a simulation progresses. The \gls{lfs} filters provide a low-cost stabilization method for all Finite Element Method-based high-order solvers to use in the cases where under-resolution and high gradients can lead to instabilities. These two efforts aim to increase the \emph{robustness} in high-order methods.

\subsection{Future Work}
\emph{Usability} also relates to an intuitive user interface. I must acknowledge the current workflow in \gls{hf} is not very intuitive: the user creates a mesh in some program, then selects parameters in an input file, then runs \gls{hf} in the command line. Only advanced users can, as of now, truly take advantage of the power of \gls{hf}. Part of future, post-graduation, efforts must include the implementation of a Graphical User Interface. An excellent example of a starting point is provided by Gmsh~\cite{geuzaine2009gmsh}.

The \gls{cmfr} schemes have variable dispersion-dissipation properties with arbitrary tunning parameters. It is still unclear how much effect each new parameter has on such properties. A Von Neumann study along the lines of that performed by Vincent et al.~\cite{vincent2011insights} would help quantify this.

The \gls{lfs} filters show great promise for their use in industrial simulations with unstructured grids. Interestingly, as shown by Asthana et al.~\cite{asthana2014}, the stronger the filter, the lower the order of the scheme; yet accuracy in high-order, filtered simulations remains higher than in low-order simulations. \gls{lfs} filters could provide a way to perform grid refinement via polynomial refinement in very large elements without worrying about instabilities. Understanding this phenomenon through grid refinement studies and analysis would be an interesting future research path.

